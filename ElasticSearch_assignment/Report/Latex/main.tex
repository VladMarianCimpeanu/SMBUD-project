\documentclass{article}
\usepackage{graphicx}
\usepackage{minted}
\usepackage{array}
\renewcommand*{\arraystretch}{2}




\title{SMBUD 2021 - Project work 3}
\author{\\Aman Gabba - 10793117 \\
Andrea Cerasani - 10680486
\\Giovanni Demasi - 10656704
\\Pasquale Dazzeo - 10562130
\\Vlad Marian Cimpeanu - 10606922}
\date{ \begin{figure}[b] \centering \includegraphics[scale=0.2]{Logo Polimi.png} \end{figure}
 }
\usepackage[dvipsnames]{xcolor}

\usepackage{ifxetex}
\usepackage{ifluatex}
\newif\ifxetexorluatex % a new conditional starts as false
\ifnum 0\ifxetex 1\fi\ifluatex 1\fi>0
   \xetexorluatextrue
\fi

\ifxetexorluatex
  \usepackage{fontspec}
\else
  \usepackage[T1]{fontenc}
  \usepackage[utf8]{inputenc}
  \usepackage[lighttt]{lmodern}
\fi

\usepackage{textcomp}
\usepackage{xcolor}
\usepackage{listings}
\usepackage{upquote}

\definecolor{keyword}{HTML}{2771a3}
\definecolor{pattern}{HTML}{b53c2f}
\definecolor{string}{HTML}{be681c}
\definecolor{relation}{HTML}{7e4894}
\definecolor{variable}{HTML}{107762}
\definecolor{comment}{HTML}{8d9094}

\lstset{
	numbers=none,
	stepnumber=1,
	numbersep=5pt,
	basicstyle=\small\ttfamily,
	keywordstyle=\color{keyword}\bfseries\ttfamily,
	commentstyle=\color{comment}\ttfamily,
	stringstyle=\color{string}\ttfamily,
	identifierstyle=,
	showstringspaces=false,
	aboveskip=3pt,
	belowskip=3pt,
	columns=flexible,
	keepspaces=true,
	breaklines=true,
	captionpos=b,
	tabsize=2,
	frame=none,
}

\lstset{upquote=true}

\lstdefinelanguage{cypher}
{
	morekeywords={
		find, sort, updateMany, insertOne, countDocuments, limit
	}
}


\newcommand{\mycdots}{\cdot\!\cdot\!\cdot}
\lstset{language=cypher,
	literate=*
	{...}{$\mycdots$}{1}
	{theta}{$\theta$}{1}
}
\lstset{escapeinside={<@}{@>}}


\begin{document}

\maketitle
\thispagestyle{empty}

\newpage

\tableofcontents

\newpage

\section{Introduction}

\subsection{Problem Specification}
The aim of this project was to design, store and query data on a NoSQL DB supporting a data analysis scenario over data about COVID-19 vaccination statistics. The purpose is that of building a comprehensive database of vaccinations.

A given vaccinations dataset has been assigned, with the purpose to pick a time interval of at least 3 months from it and, by using an ElasticSearch installation, import the data, apply the appropriate schema design choices, implement some queries aiming at exploring the data statistics and design a basic visualization dashboard of the results.

\subsection{Hypothesis}
The assumptions taken into account are the following:

\begin{itemize}


\item ...

\end{itemize}
%\newpage
%\section{ER diagram}
%The designed ER diagram contains the following entities: Person, Place, Authorized body, Certificate, Event and Sanitary operator. Nurse and Doctor have been designed as sub-entities of Sanitary Operator, Vaccine and Swab have been designed as sub-entities of Certificate and the Recovery entity is related to a swab.

%Every person, identified by an unique tax code, can face some events, like for instance a vaccination, in determinates places with unique GPS coordinates. Through an event a person can obtain a certificate with unique UCI useful for checking its valitidy. Every certification is issued by an authorized issuer and all the events are performed by prepared sanitary operators.

\hfill\break
\newpage

\section{... diagram}
...

%\hfill\break
%\begin{center}
%\includegraphics[scale=0.155]{document_diagram.png}
%\end{center}
\newpage
\section{Dataset description}
\subsection{Vaccine summary dataset}
The Dataset of the project has been downloaded from The official Italian Government Github repository at the following link :\\ \url{https://raw.githubusercontent.com/italia/covid19-opendata-vaccini/master/\\dati/somministrazioni-vaccini-latest.csv}.
\\
\hfill\break
It contains information about administered vaccines in Italy and it is made by the following fields:
\hfill\break
\begin{center}
\begin{tabular}{ |m{4cm}|m{2cm}|m{4.5cm}|}
  \hline
  \bfseries{Field} & \bfseries{Data type} & \bfseries{Description} \\
  \hline\hline
  index & integer & Record identification code\\
  \hline
  area & string & Code of the delivery region\\
    \hline
      supplier & string & Complete name of the supplier of the vaccine\\
    \hline
          administration date & datetime & Administration date of the vaccines\\
              \hline
          age group & string & Age group to which the subjects to whom the vaccine were administered belong\\
                        \hline
          male count & integer & Number of vaccinations administered to males per day, region and age group\\
                        \hline
          female count & integer & Number of vaccinations administered to females per day, region and age group\\
    \hline
  first doses & integer & Number of people administered with the first dose\\ 
    \hline
  second doses & integer & Number of people administered with the second dose\\
    \hline

\end{tabular}
\end{center}

\newpage
\begin{center}
\begin{tabular}{ |m{4cm}|m{2cm}|m{4.5cm}|}
\hline
  post infection doses & integer & Number of administrations given to subjects with previous covid-19 infection in the 3-6 month period and who, therefore, conclude the vaccination cycle with a single dose\\ 
    \hline
  booster doses & integer & Number of people administered with an additional dose/recall\\ 
    \hline
  NUTS1 code & string & European classification of NUTS territorial units: NUTS level 1\\ 
    \hline
  NUTS2 code & string & European classification of NUTS territorial units: NUTS level 2\\ 
    \hline
  ISTAT region code & integer & ISTAT code of the Region\\
    \hline
  region name & string & Standard denomination of the area (where necessary bilingual denomination)\\ 
  \hline
  \end{tabular}
\end{center}
\hfill\break
The data types written in the table are the 'original' ones, so the ones used by the dataset creator. 

The same data types have been used to implement and use the dataset in ElasticSearch because they well represent the different parameters, so no changes were needed. (CHECK??)

The dataset period taken into consideration for statistical purposes is the one that goes from the first day of the year 2021 to the last day of September 2021. (CHECK??)

\newpage
\subsection{... dataset}
We have also, as optional point of this porject, integrated our analysis with another dataset ...

\newpage
\section{Queries and Commands}
In the following chapter all the queries and commands parameters (part of the code to substitute with desired values) will be highlighted with \textbf{\color{magenta}{magenta}} bold text.

Some parameters information can be useful for different queries or commands so they are written here to avoid writing them multiple times:
\begin{itemize}
    \item ...
\end{itemize}
\subsection{Queries}
\subsubsection{Query 1}
The following query...

\begin{lstlisting}[language=cypher, label=lst:cypher-example]

CODE

\end{lstlisting}
\subsubsection{Query 2}
The following query...

\begin{lstlisting}[language=cypher, label=lst:cypher-example]

CODE

\end{lstlisting}
\subsubsection{Query 3}
The following query...

\begin{lstlisting}[language=cypher, label=lst:cypher-example]

CODE

\end{lstlisting}
\subsubsection{Query 4}
The following query...

\begin{lstlisting}[language=cypher, label=lst:cypher-example]

CODE

\end{lstlisting}
\subsubsection{Query 5}
The following query...

\begin{lstlisting}[language=cypher, label=lst:cypher-example]

CODE

\end{lstlisting}
\subsubsection{Query 6}
The following query...

\begin{lstlisting}[language=cypher, label=lst:cypher-example]

CODE

\end{lstlisting}
\subsubsection{Query 7}
The following query...

\begin{lstlisting}[language=cypher, label=lst:cypher-example]

CODE

\end{lstlisting}
\subsubsection{Query 8}
The following query...

\begin{lstlisting}[language=cypher, label=lst:cypher-example]

CODE

\end{lstlisting}
\newpage
\subsection{Commands}
\subsubsection{Command 1}
The following query...

\begin{lstlisting}[language=cypher, label=lst:cypher-example]

CODE

\end{lstlisting}
\subsubsection{Command 2}
The following query...

\begin{lstlisting}[language=cypher, label=lst:cypher-example]

CODE

\end{lstlisting}
\newpage

\section{Dashboard description}
The Kibana Dashboard has been made by...

\newpage

\section{User guide}
...

\section{Conclusion}

Some interesting conclusions can be drawn from the development of this project:

...

\section{References and Sources}
\begin{itemize}
    %\item \url{Random-italian-person package: https://pypi.org/project/random-italian-person}
    %\item \url{PyMongo package: https://docs.mongodb.com/drivers/pymongo/}
    %\item \url{Flask package: https://flask.palletsprojects.com/en/2.0.x/}
    \item Elastic Guide: \url{https://www.elastic.co/guide/index.html}
    \item Italian Government repository: \url{https://github.com/italia/covid19-opendata-vaccini}
    
    %\item Overpass turbo API: https://wiki.openstreetmap.org/wiki/Overpass\_API
\end{itemize}



\end{document}